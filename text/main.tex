
\documentclass[11pt]{article}
\usepackage{digra}
 
\usepackage[authordate,minnames=1,maxnames=2,maxbibnames=10,minbibnames=7]{biblatex-chicago}
\AtEveryBibitem{\clearfield{doi}\clearfield{urlyear}\clearfield{urlmonth}\clearfield{urlday}}
\DeclareFieldFormat[game]{title}{\mkbibemph{#1}}
\DefineBibliographyStrings{english}{%
  references = {},
}


\addbibresource{bibliography.bib}
\title{\addvspace{-2\baselineskip}Your Title goes Here: It May Carry \\
Over onto a Second Line}
\author{First Anonymous Author}
\affil{Institutional Affiliation \\
Address line 1 \\
Address line 2 \\
telephone \\
firstauthor@institution.com }

\author{Second Anonymous Author, Third Anonymous Author}
\affil{Institutional Affiliation \\
Address line 1 \\
Address line 2 \\
telephone \\
secondauthor@institution.com, thirdauthor@institution.com}
\date{\vspace{-60pt}}

\begin{document}
\pagenumbering{gobble} 
\newpage
\pagenumbering{arabic}  
\addvspace{-1\baselineskip}
\maketitle
\addvspace{-1\baselineskip}
\copyrightnotice

\section{Introduction}
MetaSteam grew from the rapid growth of Steam game libraries. Steam Sales have become a regular occurrence, as have Humble Bundles. Both drastically reduce the cost for buying numerous games in small time periods.
There are no physical limitations to game ownership on Steam. Prior to digital distribution, an owner would need physical space for game disks, manuals and so on. There is no similar requirement for digital distribution, even hard drive space is not a limitation as only a subset of owned games need to be installed at a time, the rest residing in the cloud.

Such a situation is obviously not what the Steam Library was designed for. It's design assumes relatively modest numbers of high quality games, as the case was when Steam was originally released in 2004. The purpose of Steam has grown since then. Libraries of multiple hundreds, and even thousands, of games, are becoming more prevalent.

When confronted with a list of a thousand different names, and possibly an undescriptive image, is it realistic to expect users to remember and know what each game is, where they are in it, and what it is similar to?

\section{Related Work}
The problem is starting to be recognised, with blog posts like LifeHacker's ``How to keep your overflowing steam library neatly organised'' (http://lifehacker.com/how-to-keep-your-overflowing-steam-library-neatly-organ-1352077149),
and Depressurizer (https://github.com/rallion/depressurizer), but they are only stopgaps, only assisting in use of the existing user interface. There have been design concepts (http://rendom.net/steam3/ and https://imgur.com/gallery/4dY9C/) but there have not been many attempts to actually create a valid, working alternative interface.

Something about d3 should be mentioned. 

    
 \section*{MetaSteam System Overview}

MetaSteam is separated into two parts: A Python component, and a D3 javascript based web interface. The Python part of MetaSteam serves four functions: Local Scanning, Web Scraping, Interface Hosting, and Game starting. The Web interface for MetaSteam independently visualises information about the user's games. Both components are designed to be minimal, extensible, and understandable, and available (see https://github.com/jgrey4296/metasteam).

\subsection{Python Component}
The Python component of MetaSteam is made of 5 classes: The main program (metaSteamMicro.py), server (MetaSteamHTTPServer.py), three classes to scrape information (MetaSteamProfileScraper.py, MetaSteamStoreScraper.py, and MetaSteamMultiplayerSraper.py).

\subsubsection*{Local Scanning:}
MetaSteam loads a settings json file that provides the steam username, location of the steam executable, and locations of steam libraries on hard drives. It then reads all manifest files in those locations, building up a data structure of what games are installed. This is saved to MetaSteam's own gameData json file.

\subsubsection*{Web Scraping:}
Once MetaSteam has a list of the games that are installed on the system, it starts a separately threaded Scraper that, for all installed games, parses through the Steam Store page of a game, extracting tags, developer, publisher, release date, review status, and other meta data available on the Steam Store (but not in the actual Steam Library). Such extracted data is saved into the gameData json file as well.

MetaSteam also starts a Profile Scraping thread, which downloads the user's Steam Community page that lists all their owned games. This page stores such data as json internally, so it is parsed and added to the gameData json file.

There is also a Scraper class to lookup a game's Steam Community page, and extract the number of players Steam registers as playing the game currently. This is used for assessing multiplayer activity for games in one of the web visualisations.


\subsubsection*{Interface Hosting:}
Separately from the scraping threads, MetaSteam has a simple HTTP server thread that opens the user's web browser to a localhost page, and serves up the web interface. The gameData json is one of the files served, allowing the web interface access to all of the scraped information on installed, and profile, games.

\subsubsection*{Game Starting:}
The HTTP server is able to respond to particular PUSH messages from the web interface, which calls the command line interface of steam, allowing the web interface to start games directly instead of the user needing to switch to Steam.

\subsection{Web Interface and Javascript Component}
The Web component of MetaSteam is a simple webpage that uses the d3 library. It consists of the MetaSteamHub.js file, where visualisations are imported and registered to a button (two lines are edited for a new visualisation to be added), and then individual modules for each visualisation (CirclePack.js etc), where three methods need to be implmented: draw, registerData, and cleanup.

\subsubsection*{Web Interface:}
The web interface provides a variety of visualisations for the gameData json file that the python component creates and updates. The web page is comprised of a main MetaSteamHub javascript object that maintains overall state, and draws buttons to access each registered visualisation. Each visualisation meanwhile is an object that implements three methods: A registerData method (for transforming the gameData object to a useable representation for the visualisation), a draw method, and a cleanup method (to remove anything drawn when a different visualisation is selected.

The visualisations currently written and being refined are the following: Two Circle Pack views, a Timeline view, a Multiplayer population view, Genre based Pie chart view, Update versus played view, Calendar View, Random view, and a Cooccurrence Matrix View of Tags.

\subsubsection*{Circle Pack Visualisations}
There are two circle pack visualisations. The first groups games into tags ('FPS','RPG' etc, which have been scraped from each game's Steam Store Page), which a user can select to then view games visualised by the number of hours played for each game (scraped from the user's Steam Profile Page).

The second circle pack visualisation groups games by Developer and Publisher, both extracted from the Steam Store page for each game.

\subsubsection*{Timeline visualisation}
The timeline visualisation presents a simple bar chart, with the X axis displaying release date of a game, Y Axis displaying number of hours played for each game, and highlighting over a game displaying when the game was last played.

\subsubsection*{Multiplayer population visualisation}
The Multiplayer population visualisation lists all games that have a tag passing the regex ``/[Mm]ulti[- ]?[Pp]layer/ ``. This list is passed back to the python server as a list of appids, which scrapes each game's Steam Community Page for the number of players in game currently, which is passed back to the web page for display. Games are sorted by the size of the population.

\subsubsection*{Genre Pie Chart visualisation}
The Genre/Tag pie chart visualisation aggregates the number of hours played of all games in each tag category, displaying the top tag/hours played as a Pie/Donut chart.

\subsubsection*{Update Versus Played visualisation}
The update Versus Played visualisation uses the 'last updated' field and 'last played' fields from parsed manifest files of installed games to display games that have been updated more recently than played, as a list. This is useful for early access games where updates add non-trivial changes which users may be waiting for.

\subsubsection*{Calendar Visualisation}
The Calendar visualisation currently generates sample play history data, inventing x<6 hour play sessions from the release date of the game to the current date, to be able to visualise a calendar from 2004 (Steam's release date) to the current date, as an example of how a more detailed play history view could work. Each year is displayable, with each day separately viewable, which displays any game that has been marked as having a play session on that date.

This view is currently using made up data, as Steam does not record or track more than 'last played' and 'total hours played' data. MetaSteam however has the potential, if the user starts and stops games through the MetaSteam Web Interface. 

\subsubsection*{Random Game View}
This view is a simple button that will display the information for a random installed game. The view itself is accessible from all visualisations if a user clicks on a specific game. It is from this specific game view that a user can start a game.

\subsubsection*{Tag Co-Occurrence Matrix visualisation}
The Tag co-occurrence matrix shows a very large matrix of all scraped tags, marking which tags occur together on at least one game. This is currently very large and slow to draw, but can show interesting patterns.


\section{Questionnaire and user study} 
A Questionnaire was posted to Reddit in the r/pcgaming and r/steam subreddits, to collect initial data on user views of their experience with Steam. This resulted in 300+ responses, which were filtered by Operating System, and number of games owned, to determine potential users to test MetaSteam. In total, around 25 users were contacted to try the MetaSteam prototype, of which 5 or 6 have responded and are currently using the system.

\section{Interview}



\section*{ENDNOTES}
\addvspace{-1\baselineskip} %%I can't find another way to reduce the size of the gap for the end notes
\theendnotes

\section*{BIBLIOGRAPHY}
\printbibliography
\end{document}
